\documentclass[UTF8]{ctexart}

\usepackage{algorithm}
\usepackage{algorithmic}
\usepackage{amsmath,amssymb}
\usepackage{booktabs}
\usepackage{geometry}
\usepackage{tikz}
\usepackage{color}
\geometry{a4paper,scale=0.7}

\begin{document}
    SA22225226 李青航

    \noindent\textbf{580}

    图1最大流最小割9

    图2最大流最小割12

    图3最大流最小割9

    ~\\
    \noindent\textbf{599}

    我们有
$$
P(C_1,x)=0
$$
$$P(C_2,x)=P(P_2,x)-P(C_1,x)=P(P_2,x)-0=P(P_2,x)$$
$$P(C_3,x)=P(P_3,x)-P(C_2,x)=P(P_3,x)-P(P_2,x)
$$
$$P(C_4,x)=P(P_4,x)-P(P_3,x)+P(P_2,x)$$
$$P(C_5,x)=P(P_5,x)-P(P_4,x)+P(P_3,x)-P(P_2,x)
$$

推理得
\begin{equation*}
    \begin{aligned}
        P(C_n,x)
        &=P(P_n,x)-P(P_{n-1},x)+\cdots+(-1)^nP(P_2,x)\\
        &=x(x-1)^{n-1}-x(x-1)^{n-2}+\cdots+(-1)^nx(x-1)\\
        &=\frac{x(x-1)^{n-1}+(-1)^nx}{1+(x-1)^{-1}}\\
        &=(x-1)^n+(-1)^n(x-1)
    \end{aligned}
\end{equation*}

    ~\\
    \noindent\textbf{645}

    \textbf{i.}
    \begin{equation*}
        f\circ g=
        \begin{pmatrix}
            1 & 2 & 3 & 4 & 5 & 6  \\
            2 & 5 & 3 & 4 & 1 & 6  \\
        \end{pmatrix}
    \end{equation*}

    \begin{equation*}
        g\circ f=
        \begin{pmatrix}
            1 & 2 & 3 & 4 & 5 & 6  \\
            1 & 2 & 5 & 3 & 4 & 6  \\
        \end{pmatrix}
    \end{equation*}

    \textbf{ii.}

    \begin{equation*}
        f^{-1}=
        \begin{pmatrix}
            1 & 2 & 3 & 4 & 5 & 6  \\
            4 & 3 & 6 & 2 & 5 & 1  \\
        \end{pmatrix}
    \end{equation*}

    \begin{equation*}
        g^{-1}=
        \begin{pmatrix}
            1 & 2 & 3 & 4 & 5 & 6  \\
            6 & 4 & 1 & 5 & 2 & 3  \\
        \end{pmatrix}
    \end{equation*}

    \textbf{iii.}

    \begin{equation*}
        f^{2}=
        \begin{pmatrix}
            1 & 2 & 3 & 4 & 5 & 6  \\
            3 & 1 & 4 & 6 & 5 & 2  \\
        \end{pmatrix}
    \end{equation*}

    \begin{equation*}
        f^{5}=
        \begin{pmatrix}
            1 & 2 & 3 & 4 & 5 & 6  \\
            1 & 2 & 3 & 4 & 5 & 6  \\
        \end{pmatrix}
    \end{equation*}

    \textbf{iv.}

    \begin{equation*}
        f\circ g\circ f=
        \begin{pmatrix}
            1 & 2 & 3 & 4 & 5 & 6  \\
            6 & 4 & 5 & 2 & 1 & 3  \\
        \end{pmatrix}
    \end{equation*}

    \textbf{v.}

    \begin{equation*}
        g^3=f\circ g^3\circ f^{-1}=
        \begin{pmatrix}
            1 & 2 & 3 & 4 & 5 & 6  \\
            1 & 2 & 5 & 3 & 4 & 6  \\
        \end{pmatrix}
    \end{equation*}

    ~\\
    \noindent\textbf{657}

    \textbf{i.}

    \begin{equation*}
        f\ast \mathbf{c}=(R,B,R,B,R,R)
    \end{equation*}

    \textbf{ii.}

    \begin{equation*}
        f^{-1}\ast \mathbf{c}=(R,R,B,R,R,B)
    \end{equation*}
    
    \textbf{iii.}

    \begin{equation*}
        g\ast \mathbf{c}=(R,R,R,R,B,B)
    \end{equation*}

    \textbf{iv.}

    \begin{equation*}
        (g\circ f)\ast \mathbf{c}=(R,B,R,R,B,R)
    \end{equation*}

    \begin{equation*}
        (f\circ g)\ast \mathbf{c}=(R,R,B,R,B,R)
    \end{equation*}

    \textbf{v.}
    \begin{equation*}
        (g^2\circ f)\ast \mathbf{c}=(R,R,R,B,B,R)
    \end{equation*}

    ~\\
    \noindent\textbf{662}
    \begin{equation*}
        \frac{p^4+2p^3p^2+2p}{8}
    \end{equation*}

    ~\\
    \noindent\textbf{676}

    \begin{equation*}
        \begin{matrix}
            \rho^0 \quad 6\quad  729\\ \rho^1\quad 1 \quad 3 \\ \rho^2\quad 2 \quad 9\\ \rho^3 \quad 3\quad  27 
\\ \rho^4 \quad2 \quad 9\\ \rho^5\quad 1 \quad 3 \\ \sigma_i \quad 3 \quad 27\\ \tau_i\quad 4\quad 81
        \end{matrix}
    \end{equation*}

    \begin{equation*}
        \frac{1}{|G|}\sum 3^{c(g)}=\frac{1104}{12}=92
    \end{equation*}




\end{document}