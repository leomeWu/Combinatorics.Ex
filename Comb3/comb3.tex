\documentclass[UTF8]{ctexart}
\usepackage{algorithm}
\usepackage{algorithmic}
\usepackage{amsmath,amssymb}
\usepackage{booktabs}


\renewcommand{\algorithmicrequire}{ \textbf{Input:}} %Use Input in the format of Algorithm
\renewcommand{\algorithmicensure}{ \textbf{Output:}} %UseOutput in the format of Algorithm
% 参考:https://blog.csdn.net/jzwong/article/details/52399112

\begin{document}

李青航 SA22225226\\

\noindent\textbf{31}\\

穷举得,11种覆盖\\


\noindent\textbf{38}\\

用Loubere's method方法构建7阶幻方,见表1
\begin{table}[h]
\centering
\caption{magic square of order 7}
\begin{tabular}{|ccccccc|} 
\toprule
\multicolumn{1}{|c}{30} & 39 & 48 & 1  & 10 & 19 & 28  \\
38                      & 47 & 7  & 9  & 18 & 27 & 29  \\
46                      & 6  & 8  & 17 & 26 & 35 & 37  \\
5                       & 14 & 16 & 25 & 34 & 36 & 45  \\
13                      & 15 & 24 & 33 & 42 & 44 & 4   \\
21                      & 23 & 32 & 41 & 43 & 3  & 12  \\
22                      & 31 & 40 & 49 & 2  & 11 & 20  \\
\bottomrule
\end{tabular}
\end{table}

\noindent\textbf{54}\\

$$A\rightarrow C \rightarrow D\rightarrow E \rightarrow B$$
$$A\rightarrow C \rightarrow D \rightarrow F \rightarrow B$$
$$A\rightarrow D \rightarrow E \rightarrow B$$
$$A\rightarrow D \rightarrow F \rightarrow B$$
$$A\rightarrow E\rightarrow B$$



\noindent\textbf{57}\\

这个游戏不平衡,见表2,第4、3、1位不平衡,玩家I从数量为30的堆中拿26个石头(16+8+2),使得平衡,就能使玩家I赢
\begin{table}[h]
\centering
\caption{5-heap Nim}
\begin{tabular}{c|cccccc} 
\toprule
\multicolumn{1}{l}{} & 2\^{}5 & 2\^{}4 & 2\^{}3 & 2\^{}2 & 2\^{}1 & 2\^{}0  \\ 
\hline
10                   & 0      & 0      & 1      & 0      & 1      & 0       \\
20                   & 0      & 1      & 0      & 1      & 0      & 0       \\
30                   & 0      & 1      & 1      & 1      & 1      & 0       \\
40                   & 1      & 0      & 1      & 0      & 0      & 0       \\
50                   & 1      & 1      & 0      & 0      & 1      & 0       \\
\bottomrule
\end{tabular}
\end{table}

\noindent\textbf{67}\\

记$b_i$为大师在第$i$天所下的棋局数,且有$b_i\ge 1$ ,所以有两组序列\\
\begin{equation}
\{b_1+b_2+b_3+...+b_i\}, 1\le i \le 77
\end{equation}

\begin{equation}
\{b_0+b_1++b_2...+b_j+k\}, 0\le j \le 76 , b_0=0
\end{equation}

每组序列都满足单增且大小都为77,序列(1)中没有重复元素,序列(2)中也没有重复元素。

序列(1)的上界为$12 \times 11 = 132$,序列(2)的上界S满足$S < b_1+b_2 + \cdots + b_{77} + k \le 12 \times 11 + 22 = 154$, 同时S为整数,可以得到$S \le 153$。

综上,序列(1)和序列(2)的上界为153,也即最多有153个不同的数。而序列中一共有154个数, 由鸽巢原理序列(1)中至少有一个数$b_1+b_2 + \cdots + b_y$ 与序列(2)中的某个数$b_1+b_2 + \cdots + b_x+k$相同,满足,

$$b_1+b_2+b_3+...+b_y=b_1+b_2+b_3+...+b_x+k$$

可以求出存在
$$k=b_{x+1}+...+b_y$$

所以有$x+1$到$y$天,大师下了刚好22盘。

\end{document}