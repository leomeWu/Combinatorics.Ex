\documentclass[UTF8]{ctexart}
\usepackage{algorithm}
\usepackage{algorithmic}
\usepackage{amsmath,amssymb}
\usepackage{booktabs}
\usepackage{geometry}

\renewcommand{\algorithmicrequire}{ \textbf{Input:}} %Use Input in the format of Algorithm
\renewcommand{\algorithmicensure}{ \textbf{Output:}} %UseOutput in the format of Algorithm
% 参考:https://blog.csdn.net/jzwong/article/details/52399112

\geometry{a4paper,scale=0.7}

\begin{document}

SA22225226 李青航

\noindent \textbf{124}

第一步,男人有$P(5,5)$种排列,又因为是圆形桌子,除5

第二步,女人在两个男人中间插空坐,坐在男人中间,又有$P(5,5)$种排列

第三步,狗在10个人之间插空坐,有10种

所以总的有$P(5,5)\div 5\times P(5,5) \times 10$种

~\\
\noindent \textbf{126}

总共$3+4+5=12$个元素,选11个排列,分3种情况,直接套多重集合排列公式2.4.2

情况1:$\{2\cdot a, 4\cdot b,5 \cdot c\}$\\
$$
\dfrac{11!}{2!\cdot 4!\cdot  5!}
$$

情况2:$\{3\cdot a,3\cdot b, 5\cdot c\}$
$$
\dfrac{11!}{3!\cdot 3!\cdot  5!}
$$

情况3:$\{3\cdot a,4\cdot b, 4\cdot c\}$
$$
\dfrac{11!}{3!\cdot 4!\cdot  4!}
$$

~\\
\noindent \textbf{132}

记$y_1 = x_1 - 2, y_2 = x_2, y_3 = x_3 + 5, y_4 = x_4 - 8$,问题转化为求方程$y_1 + y_2 +y_3 +y_4 = 25$的非负整数解个数,直接套多重集合组合公式2.5.1的内容,共
$$
{{25+4-1}\choose{3}}={{25+4-1}\choose {25}}
$$

~\\
\noindent \textbf{137}

不带1的情况:
$$
{(r-1)+k-1}\choose{r-1}
$$

带1的情况:
$$
{(r-1)+(k-1)-1}\choose{r-1}
$$

总的情况为上述两式相加

~\\
\noindent \textbf{149}

经过统计,各个字母的个数如下:\\
\begin{center}
\noindent('O', 9), ('I', 6), ('C', 6), ('N', 4), ('S', 4), ('L', 3), ('P', 2),\\
 ('U', 2), ('M', 2), ('R', 2), ('A', 2), ('E', 1), ('T', 1), ('V', 1)\\
\end{center}
套用多重集合排列公式,排列共有
$$
\dfrac{45!}{9!\cdot (6!)^2 \cdot (4!)^2\cdot 3! \cdot (2!)^5}
$$

~\\
\noindent \textbf{151}

$$
\overrightarrow{4}\overleftarrow 8\overrightarrow 3\overleftarrow 1 \overrightarrow 6\overleftarrow 7 \overleftarrow 2\overrightarrow 5
$$

根据规则,如果一个整数$k$,箭头指向一个与其相邻但比它小的整数,那么这个整数$k$就是可以移动的。

所以8,3,7可以移动

~\\
\noindent \textbf{156}

\textbf{i. }使用算法1
\begin{table}[h]
\centering
\begin{tabular}{l}
8         \\
87        \\
867       \\
8657      \\
48657     \\
486573    \\
4865723   \\
48165723 
\end{tabular}
\end{table}

\textbf{ii. }使用算法2

\begin{table}[h]
\centering
\begin{tabular}{llllllll}
  &   &   &   &   &   & 1 &    \\
  &   &   &   &   &   & 1 & 2  \\
  & 3 &   &   &   &   & 1 & 2  \\
  & 3 &   &   &   & 4 & 1 & 2  \\
  & 3 &   & 5 &   & 4 & 1 & 2  \\
  & 3 & 6 & 5 &   & 4 & 1 & 2  \\
7 & 3 & 6 & 5 &   & 4 & 1 & 2  \\
7 & 3 & 6 & 5 & 8 & 4 & 1 & 2 
\end{tabular}
\end{table}


\end{document}